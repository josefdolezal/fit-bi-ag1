\section{Řazení}

  \subsection{Vlastnosti}

    \subsubsection{Varianty řazení}
      Řazení probíhá buď offline (nejprve přijdou data, následně se řadí) nebo online (úkoly chodí průběžně během zpracování).

    \subsubsection{Paměťová náročnost}
      Algoritmy jsou buď in-place (využívají kromě počáteční paměti jen konstantní paměť navíc). Out-of-place algoritmy řadí prvky v nově naalokované paměti.

    \subsubsection{Stabilita}
      Určuje, jestli navzájem si rovné prvky jsou v seřazené podobě ve stejném pořadí (stabilní) nebo proházené (nestabilní).

    \subsubsection{Citlivost}
      Udává, zda algoritmus má stejnou složitost pro libovolný vstup (datově necitlivý) nebo se složitost mění v závislosti na vstupu (datově citlivý).

  \subsection{Algoritmy řazení}

    \subsubsection{BubbleSort}
      BubbleSort funguje na principu prohazování dvojic, kde vždy po porovnání dvojci buď prohodí nebo ponechá ve stejném pořadí a posune se o prvek dál.
      Po prvním průchodu je na konci největší prvek napravo a dále se řadí $n - 1$ prvků.
      Algoritmus se zastaví, pokud v jednom běhu zleva doprava neproběhlo žádné prohození.

      \begin{itemize}
        \item Vhodný na částečně seřazenou posloupnost.
        \item Seřadí prvky v čase $ \textrm O(n\textsuperscript{2})$.
        \item Stabilní, in-place, datově citlivý.
      \end{itemize}

    \subsubsection{ShakerSort}
      Nadstavba nad BubbleSort. Prochází oběma směry, tedy po první průchodu (zleva doprava) je vpravo největší prvek. Po druhém průchodu (zprava doleva) je nalevo nejmenší prvek.

      \begin{itemize}
        \item Řeší problém pomalého posunu malých prvů směrem doleva, snižuje tedy počet porovnání.
        \item Asymptoticky stále $ \textrm O(n\textsuperscript{2})$.
        \item Stabilní, in-place, datově citlivý.
      \end{itemize}

    \subsubsection{SelectSort}
      Posloupnost se rozdělí na seřazenou (zpočátku prázdná) a neseřazenou. Provádí se opakovaný výběr nejmenšího prvku, který se vymění s prvním prvkem neseřazené části.

      \begin{itemize}
        \item Asymptoticky $ \textrm O(n\textsuperscript{2})$.
        \item Nestabilní, in-place, datově necitlivý.
      \end{itemize}

    \subsubsection{InsertSort}
      Řazení vkládáním. Posloupnost rozdělíme na seřazenou (první prvek posloupnosti) a neseřazenou ($n - 1$ prvků).
      V prvním kroce vybereme následující prvek a porovnáme ho s prvním a vložíme na správné místo.
      V dalším kroce bereme následující prvek a porovnáme ho s předchozími, vložíme na správné místo a zbytek podle potřeby posuneme.
      Takto postupujeme dokud není neseřazená část prázdná.

      \begin{itemize}
        \item Stabilní, in-place, datově citlivý.
      \end{itemize}

    \subsubsection{HeapSort}
      Viz. HeapSort, kapitola Haldy.
