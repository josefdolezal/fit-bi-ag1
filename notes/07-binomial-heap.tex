\section{Binomiální halda}

  \begin{description}
    \item[Binomiální strom řádu $k$] Uspořádaný zakořeněný strom, pro který platí:
    \begin{itemize}
      \item B\textsubscript{0} je tvořen pouze kořenem.
      \item B\textsubscript{k} lze získat pomocí stromu B\textsubscript{k-1}, kterému připojíme jako nejpravějšího syna strom B\textsubscript{k}.
    \end{itemize}
    \item[Binomiální halda] Složení souboru binomiálních stromů. Každý strom dodržuje haldové uspořádání. V uspořádání se žádný strom se stejným řádem nevyskytuje dvakrát. Soubor je uspořádán vzestupně.
  \end{description}

  \subsection{Vlastnosti binomiálních stromů}
    \begin{itemize}
      \item Počet hladin stromu B\textsubscript{k} je $k+1$.
      \item Stupeň kořene stromu B\textsubscript{k} je $k$.
      \item Počet vrcholů stromu B\textsubscript{k} je $2^k$.
      \item Binomiální strom s $n$ vrcholy má hloubku $log \, n$ a počet synů kořene také $log \, n$.
    \end{itemize}

    \subsection{Vlastnosti binomiální haldy}
      \begin{itemize}
        \item Binomiální strom B\textsubscript{k} se v souborů stromů vyskytuje tehdy, je li v dvojkovém zápisu čísla $n$ nastavený $k$-tý nejnižší bit na 1.
        \item Halda se skládá z $log \, n$ binomiálních stromů.
      \end{itemize}

    \subsection{Sloučení hald}
      Sloučení probíhá obdobně jako při sčítání v binární sčítačce, tedy slučuji stromy se stejným stupněm a udržuji si přenos (aby nevzniklo více stromů se stejným stupněm $k$).

    \subsection{Vložení prvku}
      Provádí se pomocí slučování hald (slučuji s jednoprvkovou haldou).

    \subsection{Odstranění minima}
      Minimum nalezneme průchedem kořenů jednotlivých stromů. Od minima odtrhneme všechny potomky (a jejich podstromy) a sloučíme je do nové haldy.
      Tuto novou haldu sloučíme s původní.
