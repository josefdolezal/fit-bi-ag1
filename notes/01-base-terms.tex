\section{Základní pojmy}

  \begin{description}
    \item[Zdroj] Uzel, do kterého nevede žádná hrana
    \item[Úplný graf] Graf, ve kterém pro libovolné různé dva vrcholy z množiny vrcholů existuje hrana.
    \item[Úplný bipartitní graf] Graf, který lze rozdělit na dvě partity takové, že vrcholy jedné pratity nemají nemají vzájemně žádnou hranu, ale mají hranu s každým vrcholem druhé partity.
    \item[Úplný k-partitní graf] Viz. úplný bipartitní graf.
    \item[Cesta P\textsubscript{m}] Posloupnost vrcholů s \emph{m} hranami, pro kterou platí, že v grafu existuje hrana z daného vrcholu do následníka. Vrcholy ani hrany se nesmí opakovat. Délkou cesty se rozumí počet hran.
    \item[Stupeň vrcholu deg\textsubscript{G}(v)] Číslo označující počet hran při vrcholu \emph{v}.
    \item[Otevřené okolí vrcholu \emph{v}] Množina všech sousedů vrcholu \emph{v}.
    \item[Uzavřené okolí vrcholu \emph{v}] Množina všech sousedů vrcholu \emph{v} včetně vrcholu \emph{v}.
    \item[Regulární graf] Graf je r-regulární, pokud pro všechny vrcholy platí, že mají stupeň r. Graf je regulární, pokud je r-regulární pro nějaké r.
    \item[Izolovaný vrchol] Vrchol stupně 0.
  \end{description}
