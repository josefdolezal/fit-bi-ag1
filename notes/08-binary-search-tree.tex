\section{Binární vyhledávací strom}

  \begin{description}
    \item[Binární vyhledávací strom] Binární strom, ve kterém v každém uzlu platí, že levý syn je menší a pravý větší než otec.
  \end{description}

  \subsection{Nalezení minima}
    \begin{enumerate}
      \item Je-li levý syn prázdný, vrať sebe.
      \item (nebo) Vrať minimum z levého podsyna.
    \end{enumerate}

  \subsection{Vložení prvku}
    \begin{enumerate}
      \item Je-li kořen $v$ prázdný, vytvoř nový uzel a vrať ukazatel na $v$.
      \item (nebo) Je-li hodnota menší než kořen $v$, vlož hodnotu do levého podstromu a vrať výsledek.
      \item (nebo) Je-li hodnota větší než kořen $v$, vlož hodnotu do pravého podstromu a vrať výsledek.
      \item (nebo) Hodnota už ve stromu je, vrať ukazatel na $v$.
    \end{enumerate}

  \subsection{Odstranění prvku}
    Při odstranění můžou nastat 3 různé situace:
    \begin{enumerate}
      \item Vrchol nemá žádného syna, lze ho odebrat.
      \item Vrchol má jednoho syna, lze ho synem nahradit.
      \item Vrchol má oba syny, je nutné nalézt následníka a tím ho nahradit, následník má zcela jistě nejvýše jednoho syna.
    \end{enumerate}

  \subsection{Časová složitost, vyvážení}
  Časová složitost operací je rovna počtu hladin $\textrm O(h(v)).$ Přičemž hloubka stromu je minimálně $(log|T(v)|)$,
  nejhůře pak $\textrm O(|T(v)|)$.

  \begin{description}
    \item[Dokonale vyvážený BVS] Strom, pro který v každém jeho vrcholu $v$ platí, že $||L(v)| - |R(v)|| ≤ 1$.
  \end{description}

  Je-li strom dokonale vyvážený, mají operace nad ním složitost $\textrm O(log \, n)$, vzhledem k hloubce $\floor{log \, n}$.
