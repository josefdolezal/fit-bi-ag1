\section{Rekurze, metoda rozděl a panuj}
  \begin{description}
    \item[Rekurzivní algoritmus] Postup řešení problému, kdy se nad vstupními daty opakuje stejný postup pro určité části dat. Staví na stejném řešení stejného podproblému s menším rozsahem.
  \end{description}

  \subsection{MergeSort}
    Pomocí rekurze půlí intervaly, které má řadit dokud intervaly nejsou jednoprvkové. Při návratu z rekurze tyto jednoprvkové seřazené posloupnosti "slévá".

  \subsection{Rekurze}
    \begin{description}
      \item[Koncová] Rekurzivní volání je posledním příkazem algoritmu, po kterém se už neprovádějí žádné další operace.
      \item[Lineární] V těle algoritmu je jen jedno volání rekurze (nebo dvě ale disjunktní - provede se pouze jedno).
      \item[Stromová] V těle algoritmu jsou alespoň dvě rekurzivní volání. Obvykle algoritmy rozděl a panuj: $F(n) = F(n-1) + F(n-2)$.
    \end{description}
