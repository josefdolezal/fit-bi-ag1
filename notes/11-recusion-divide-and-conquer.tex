\section{Rekurze, metoda rozděl a panuj}
  \begin{description}
    \item[Rekurzivní algoritmus] Postup řešení problému, kdy se nad vstupními daty opakuje stejný postup pro určité části dat. Staví na stejném řešení stejného podproblému s menším rozsahem.
  \end{description}

  \subsection{MergeSort}
    Pomocí rekurze půlí intervaly, které má řadit dokud intervaly nejsou jednoprvkové. Při návratu z rekurze tyto jednoprvkové seřazené posloupnosti "slévá".

  \subsection{Rekurze}
    \begin{description}
      \item[Koncová] Rekurzivní volání je posledním příkazem algoritmu, po kterém se už neprovádějí žádné další operace.
      \item[Lineární] V těle algoritmu je jen jedno volání rekurze (nebo dvě ale disjunktní - provede se pouze jedno).
      \item[Stromová] V těle algoritmu jsou alespoň dvě rekurzivní volání. Obvykle algoritmy rozděl a panuj: $F(n) = F(n-1) + F(n-2)$.
    \end{description}

  \subsection{QuickSort}
    QuickSort je algoritmus, který k řazení využívá takzvaného pivota. Pomocí pivota rozdělí posloupnost na tři části:
    \begin{enumerate}
      \item Levá - Prvky menší než pivot
      \item Střední - Prvky rovné pivotu
      \item Pravá - Prvky větší než pivot
    \end{enumerate}

    Následně algoritmus pustí sám sebe na Levou a Pravou část. Výsledky se na konci spojí a vznikne seřazená posloupnost.

  \subsection{CountingSort}
    Algoritmus, u kterého je předem nutné znát minimum, maximum a celkový počet prvků:
    \begin{enumerate}
      \item Vytvoř dvě pole o velikosti $n$.
      \item Projdi vstupní data a spočítej kolikrát se každý prvek v poli nachází.
      \item Počáteční pozici nejmenšího prvku nastav na $1$, následně každou další pozici dopočítej jako pozice předchozího prvku + počet předchozího prvku.
      \item Vypiš prvky tak, že projdeš vstupní pole a do výstupního pole vypíšeš prvek na pozici, kterou udává tabulka pozic, následně pozici v tabulce pozic inkrementuj.
    \end{enumerate}
