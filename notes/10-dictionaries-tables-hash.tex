\section{Slovníky, tabulky a hashování}
  \subsection{Hashování}
    \begin{description}
      \item[Hashování] Pro množinu klíčů zvolíme konečnou množinu umístění (hashovací tabulka) a hashovací funkci, která každému klíči přiřadí právě jednu přihrádku.
      Cílem je zvolit takovou velikost a takovou funkci, aby se počet kolizí minimalizoval.
      \item[Kolize] Protože hashovací tabulka má menší kapacitu než je velikost množiny klíčů, může se stát, že několik klíčů se po použití hashovací funkce zobrazí na jedno umístění.
      Tento jev se nazývá kolize.
    \end{description}

    \subsubsection{Kolize}
      \begin{itemize}
        \item Metoda řetízků - každé umístění v tabulce je buď prázdné, nebo v něm začíná spojový seznam uložených prvků.
      \end{itemize}

    \subsubsection{Hashovací funkce}
      Ideální hashovací funkci (bez kolizí) nelze sestrojit, proto se používají funkce které se chovají "náhodně".
      \begin{itemize}
        \item Lineární kongruence $x \to ax \, mod \, m$
          \begin{itemize}
            \item Konstanta $m$ bývá prvočíslo a $a$ je dostatečně velká konstanta nesoudělná s $m$ (typicky $0.618m$).
          \end{itemize}
        \item Vyšší bity součinu
        \item Skalární součin
        \item Polynom
      \end{itemize}

      \subsubsection{Složitost}
        Pro ideální průběh hashování vyždujeme následující vlastnosti
        \begin{enumerate}
          \item Výpočet hashovací funkce je rychlý $\textrm O(1)$.
          \item Funkce rozděluje univerzum rovnoměrně.
          \item Vstupní data z univerza jsou na vstupu rozdělena rovnoměrně.
        \end{enumerate}

        Uvažujeme-li hashovací tabulku velikosti $m$ s $n$ prvky a funkci $h$ splňující nároky.
        Potom je střední hodnota hledání/mazání/vkládání $O(n/m)$.

      \subsubsection{Hashování s otevřenou adresací}
        Použije se pole s $m$ přihrádkami, kde se do každé přihrádky vejde pouze jeden prvek.
        V případě, že je místo obsazené, zkouší se náhradní tak dlouho, dokud není nalezené volné místo.
        Hashovací funkce tedy ke každému prvku $k$ z univerza vyhledávací posloupnost.
        Ta určuje v jakém pořadí se budou náhradní umístění hledat, funkce $h$ je tedy $h(k, i)$.

        \textbf{Mazání}

          Pro mazání se používá metoda náhrobků. Pokud je prvek vymazený, označí se umístění náhrobkem a jiný prvek už tam nemůže být vložen.
          Pokud by se označil jako prázdný, byl by problém s vyhledáváním: to by mohlo předčasně skončit.
          Jakmile je zaplněný určitý počet prvků (např. $m/4$), tabulka se celá přehashuje.

        \textbf{Funkce}
        
          \begin{description}
            \item[Lineární přidávání] Prohledávací posloupnost je dána funkcí $h(k, i) = (f(k) + i) \, mod \, m$ kde $f(k)$ je obyčená hashovací funkce. Využívá po sobě jdoucí přihrádky, ale tim vytváří souvislé bloky.
            \item[Dvojité hashování] Posloupnost je dána funkcí $h(k, i) = (f(k) + i * g(k)) \, mod \, m$ kde $f(k)$ i $h(k)$ jsou hashovací funce, $m$ je prvočíslo a $i$ je počet neúspěšných pokusů.
          \end{description}
