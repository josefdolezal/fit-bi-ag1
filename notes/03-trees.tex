\section{Stromy}
  \begin{description}
    \item[Strom] Graf, který je souvislý a zároveň neobsahuje cyklus.
    \item[Les] Graf, který neobsahuje žádnou kružnici (jeho jednotlivé komponenty jsou stromy).
    \item[List] Vrchol se stupněm 1.
  \end{description}

  \subsection{Listy}
    \begin{itemize}
      \item Každý strom obsahující alespoň dva dva vrcholy obsahuje alespoň dva listy.
      \item Je-li G graf o alespoň 2 vrcholech a \emph{v} je list, pak $G - v$ je také strom.
    \end{itemize}

  \subsection{Vlastnosti}
    \begin{itemize}
      \item Pro každé dva vrcholy existuje právě jedna cesta mezi nimi.
      \item Strom je souvislý a vynecháním libolné hrany bude nesouvislý.
      \item Počet vrcholů je o 1 větší než počet hran, tedy $|V| = |E| + 1$
    \end{itemize}

\section{Vzdálenost v grafech, topologické uspořádání}

  \subsection{Kostra grafu}
    \begin{itemize}
      \item Podgraf souvislého grafu \emph{G}, který obsahuje všechny vrcholy původního grafu a přitom je stromem.
      \item Nesouvislé grafy nemají kostru.
      \item Souvislý graf s kružnicemi má více koster.
      \item Nějakou kostru lze nalézt algoritmem DFS. Složitost nalezení je $\textrm O(|V| + |E|)$
    \end{itemize}

  \subsection{BFS - Prohledání do šířky, nejkratší cesta}
    Algoritmus prochází graf pomocí iterativního výběru sousedů, které přidává do fronty. Poté co v uzlu přidá všechny potomky do fronty, vezme další uzel z fronty a přidá jeho potomky.

    \begin{itemize}
      \item Algoritmus při průchodu nalezne nejkratší cestu z počátku \emph{s} do libovolného uzlu \emph{v}.
      \item Pomocí BFS lze najít kostru grafu (tzv. kostru do šířky).
    \end{itemize}

  \subsection{Topologické třídění}
    \begin{itemize}
      \item Orientovaný graf G je acyklickým pokud neobsahuje orientovanou kružnici.
      \item Je-li G orientovaný a acyklický, pak obsahuje zdroj i stok.
    \end{itemize}

    \begin{description}
      \item[Topologické uspořádání] Uspořádání vrcholů grafu do řady $v_1, v_2, \dots, v_n$ takové, kde pro všechny hrany platí, že hrany vedou pouze z vrcholů s menším indexem do vrcholu s menším.
    \end{description}

    Topologické uspořádání lze nalézt pomocí algoritmu DFS.
