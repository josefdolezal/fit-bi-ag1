\section{Grafy}

  \subsection{Vlastnosti grafů}
    \begin{description}
       \item[Princip sudosti] Pro každý graf platí, že součet stupňů všech vrcholů je roven dvojnásobku počtu hran. Z toho plyne, že počet vrcholů s lichého stupně sudý.
       \item[Počet hran úplného grafu] K\textsubscript{n} má \frac{n(n-1)}{2}.
       \item[Počet hran v úplnén bipartitním grafu] K\textsubscript{m,n} má \frac{mn}{2} hran.
    \end{description}

  \subsection{Podgrafy}

    \begin{description}
      \item[Podgraf] Libovolná podmnožina vrcholů a hran grafu G.
      \item[Indukovaný podgraf] Libovolná množina vrcholů grafu G a všechny hrany, které navzájem spojují vrcholy ze zvolené podmnožiny.
      \item[Klika grafu] Takový podgraf H grafu G, že H je úplný graf. Je tedy izomorfní s nějakým úplným grafem.
      \item[Nezávislá množina] Indukovaný podgraf, který nemá žádné hrany.
      \item[Souvislý graf] Graf, ve kterém pro každé dva vrcholy u a v platí, že existuje cesta z u do v. V opačném případě je nesouvislý.
      \item[Souvislá komponenta] Indukovaný podgraf H grafu G, který je souvislý a zároveň neexistuje podgraf F grafu G takový, že H je podgraf F. Tedy je největší možný souvislý podgraf.
    \end{description}

  \subsection{Souvislost grafu}

    \subsubsection{DFS - Prohledávání do hloubky}
      \begin{itemize}
        \item Z anglickégo \emph{Depth First Search}, tedy prohledávání "nejdříve do hloubky".
        \item Algoritmus se stále zanořuje, ostatní sousedy prochází až po vynoření.
        \item Tento algoritmus je vhodný na vyhledávání souvlislých komponent.
      \end{itemize}

  \subsection{Orientované grafy}

    \begin{description}
      \item[Orientovaný graf] Uspořádaná dvojice \emph{(V, E)}, kde V je neprázdná konečná množina vrcholů, E je množina orientovaných hran.
      \item[Orientovaná hrana] Uspořádaná dvojice (u, v) je dvojice různých vrcholů u,v. Vrchol u je předchůdce, v je následník.
    \end{description}

    \subsubsection{Stupeň vrcholu v orientovaném grafu}
      \begin{description}
        \item[Vstupní stupeň] Symbolem deg\textsubscript{G}\textsuperscript{+}(v) označíme počet hran grafu G, končících ve vrcholu v.
        \item[Výstupní stupeň] Symbolem deg\textsubscript{G}\textsuperscript{-}(v) označíme počet hran grafu G, vycházejících z vrcholu v.
        \item[Stupeň vrcholu] Jako stupeň vrcholu označíme deg\textsubscript{G}(v) = deg\textsubscript{G}\textsuperscript{+}(v) + deg\textsubscript{G}\textsuperscript{-}(v)
      \end{description}

    \subsubsection{Okolí stupně v orientovaném grafu}
      \begin{description}
        \item[Vstupní okolí] Množina všech vrcholů N\textsubscript{G}\textsuperscript{+}(v), ze kterých vede hrana do vrcholu v (množina předchůdců).
        \item[Výstupní okolí] Množina všech vrcholů N\textsubscript{G}\textsuperscript{-}(v), do kterých vede hrana z vrcholu v (množina následníků).
        \item[Okolí] Sjednocení N\textsubscript{G}(v)
      \end{description}

    \subsubsection{Zdroje, stoky, izolované vrcholy}
      \begin{description}
        \item[Izolovaný vrchol] Stupeň vrcholu je 0.
        \item[Zdroj] Vstupní stupeň vrcholu je 0.
        \item[Stok] Výstupní stupeň vrcholu je 0.
      \end{description}

    \subsubsection{Souvislost}
      \begin{description}
        \item[Symetrizace] Odstranění z grafu informace o orientaci hran.
        \item[Slabá souvislost] Graf je slabě souvislý, pokud je souvislá jego symetrizace (souvislý po odstranění orientace hran).
        \item[Silná souvislost] Graf je silně souvislý, pokud pro každé vrcholy u a v existuje orientované cesta z u do v a současně existuje orientovaná cesta z v do u.
      \end{description}
