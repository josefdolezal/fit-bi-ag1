\section{Minimální kostra}
  \begin{description}
    \item[Minimální kostra] Kostra, kteréa má mezi všemi kostrami nejmenší součet vah hran.
  \end{description}

  Na vyhledávání se používá Jarníkův (Primův) algoritmus.

  \section{Řezy grafu}
    \begin{description}
      \item[Řez grafu] Pokud rozdělíme vrcholy grafu na dvě disjunktní množiny $A$ a $B$, hranám majícím jeden vrchol v $A$ a jeden v $B$ řikáme elementární řez grafu.
    \end{description}

  \subsection{Jarníkův algoritmus}
    \begin{enumerate}
      \item Začíná se s jedním vrcholem a žádnou hranou.
      \item Ze všech již objevených vrcholů se vezmou v potaz všechny hrany, které vedou do neobjeveného vrcholu a z nich se vybere ta s nejmenší vahou.
      \item Nyní se algoritmus opakuje.
    \end{enumerate}

  \subsection{Kruskalův algoritmus}
    Založený na stejném principu jako Jarníkův algoritmus, tedy vybírání hran s nejmenší váhou.
    Oproti Jarníkovi ale používá metodu spojování stromů.
    Pokud přidává hranu, musí se vždy podívat, jestli spojuje dvě různé komponenty.

    \subsubsection{Implementace}
    \begin{enumerate}
      \item Polem (ke každému prvku si držíme, v jaké komponentě se nachází)
      \item Pomocné stromy (keříky) - v poli se drží keřík tak, že každý vrchol si pamatuje svého předka. Při vyhledávání se porovnají kořeny keříků, pokud jsou stejné tak se jedná o jednu komponentu. V opačném případě se mělčí připojí pod hlubší.
    \end{enumerate}
