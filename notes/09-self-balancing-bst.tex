\section{AVL - Hloubkově vyvážené stromy}
  \begin{description}
    \item[AVL strom] Hloubkově vyvážený strom, ve kterém pro kavzdý vrchol $v$ platí $|h((l(v))) - h(r(v)) ≤ 1|$.
  \end{description}

  \begin{itemize}
    \item AVL strom na $n$ vrcholech má hloubku právě $log \, n$.
  \end{itemize}

  Operace vložení a odstranění vrcholu mohou způsobit, že strom přestane být vyvážený.
  Z tohoto důvodu se v každém vrcholu udržuje znaménko vrcholu, které nabývá hodnot {-1, 0, 1}.
  Znaméko vrcholu lze spočítat jako $h(r(v)) - h(l(v))$, tedy hloubka pravého prodstromu ponížená o hloubku levého.
  Vyskytne-li se v nějakém vrcholu jiné znaménko, je potřeba strom opravit pomocí rotace.

  \subsection{Vložení prvku}
    \begin{itemize}
      \item Nový vrchol se vkládá jako list se znaménkem 0.
      \item Pomocí rekurze je potřeba přepočítat znaménka předků na cestě ke kořeni. Může nastat více případů:
        \begin{enumerate}
          \item Podmínka je porušena ve vrcholu, který je levým synem:
            \begin{enumerate}
              \item Jeho levý podstrom je vyšší - rotace LL.
              \item Jeho pravý podstrom je vyšší - rotace LR (převede na problém LL).
            \end{enumerate}
          \item Podmínka je porušena ve vrcholu, který je pravým synem:
            \begin{enumerate}
              \item Jeho pravý podstrom je vyšší - rotace RR.
              \item Jeho levý podstrom je vyšší - rotace RL (převede na problém RR).
            \end{enumerate}
        \end{enumerate}
    \end{itemize}
