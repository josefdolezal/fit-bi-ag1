\documentclass[20pt]{extarticle}
\usepackage[utf8]{inputenc}
\usepackage[czech]{babel}
\usepackage{t1enc}
\usepackage{amsfonts}
\usepackage{amsmath}
\usepackage{xcolor}

% iPhone 7 plus: 9:16
% showframe
\usepackage[paperheight=4.5in,paperwidth=8in,margin=0.5in,heightrounded]{geometry}

% increase math line height
\addtolength{\jot}{1em}

\pagenumbering{gobble}

\newcommand{\card}[2][]{
	\vspace*{\fill}

	\newpage
	\topskip0pt
	\vspace*{\fill}
		\Large #2

		\vspace{1cm}
		\normalsize #1
	\vspace*{\fill}
	\newpage

	\small \textcolor{gray}{#2 #1}
	\topskip0pt
	\vspace*{\fill}

	\normalsize
}
\newcommand{\splitline}{\noindent\rule{8cm}{0.4pt}}
\newcommand{\pair}[2]{\left(#1, #2\right)}

\begin{document}
\begin{center}

\topskip0pt
\vspace*{\fill}
\Large Josef Doležal\\[1cm]
\normalsize Veta otázky ZS 2016/7 \textsc{bi-AG1}\\
\normalsize

\card[zdroj a stok]{Acyklický orientovaný graf}
\begin{align*}
  Uspořádaná dvojice \pair{V}{G}, kde $V$ je neprázdná množina vrcholů a $E$ množina orientovaných hran taková, že neobsahuje cyklus. \\
  Zdroj - vrchol, do kterého nevede žádná hrana \\
  Stok - vrchol, ze kterého nevede žádná hrana
\end{align*}

\card{Věta o existenci zdroje v orientovaném grafu}
\begin{align*}
  Každý orientovaný graf, který neobsahuje cyklus, má alespoň jeden zdroj.
\end{align*}

\card{Věta o existenci zdroje v orientovaném grafu}
\begin{align*}
  Každý orientovaný graf, který neobsahuje cyklus, má alespoň jeden zdroj.
\end{align*}

\card{Algoritmus topologického uspořádání orientovaného grafu}
\begin{align*}
  \begin{enumerate}
    \item Zařaď do fronty všechny vrcholy se vstupním stupněm 0.
    \item (dokud není fronta prázdná) Vyber vrchol z počátku fronty
    \item - Vypiš vybraný vrchol
    \item - Pro každého následníka zkontrontroluj, jestli po odstranění hrany má vstupní stupeň 0.
    \item - Pokud má stupeň 0, přidej ho do fronty.
  \end{enumerate}
\end{align*}

\card{Topologické uspořádání orientovaného grafu}
\begin{align*}
  Topologické uspořádání orientovaného acyklického grafu $G = \pair{V}{E}$ je takové pořadí vrcholů $v1, v2, ..., vn$ grafu $G$,
  že pro každou hranu $\pair{v_i}{v_j}$ ∈ \textrm{E} platí $i \lt j$.
\end{align*}

\card{Neorientovaný graf}
\begin{align*}
  Neorientovaný graf je uspořádaná dvojice \pair{V}{E}, kde
  \begin{enumerate}
    \item $V$ je množina vrcholů
    \item $E$ je množina hran
  \end{enumerate}
  Hrana je dvouprvková podmnožina $V$.
\end{align*}

\card{Orientovaný graf}
\begin{align*}
  Orientovaný graf G je uspořádaná dvojice \pair{V}{E}, kde
  \begin{itemize}
    \item $V$ je neprázdná konečná množina vrcholů
    \item $E$ je množina orientovaných hran
  \end{itemize}
  Orientovaná hrana \pair{u}{v} ∈ $E$ je uspořádaná dvojice různých vrcholů $u$,
  $v$ ∈ $V$ . Říkáme, že $u$ je předchůdce $v$ a $v$ je následník $u$.
\end{align*}

\card{Úplný graf $K_n$}
\begin{align*}
  Úplný graf na $n$ ($n \ge 1 $) vrcholech $K_n$ je graf $\pair{v}{\binom{V}{2}}$, kde $|V| = n$.
\end{align*}

\card{Úplný bipartitní graf $K_n$}
\begin{align*}
  Nechť $n \geg 1$ a $m \geg 1$.
  Úplný bipartitní graf s $n$ vrcholy v jedné partitě a $m$ vrcholy v druhé
  partitě $K_n,m$ je graf $(A \cup B, {{a,b} | a \in A,b \inB})$, kde
  $A \cap B = \emptyset, |A| = n a |B| = m$.
\end{align*}

\card{Kružnice $C_n$}
\begin{align*}
  Nechť $n \geg 1$. Kružnice délky $n$ (s $n$ vrcholy) je graf
  $({1, \dots, n}, {{i, i+1} | i \in {1, \dots, n-1}} \cup {{1, n}})$.
\end{align*}

\card{Cesta $P_m$}
\begin{align*}
  Nechť $m \geg 0$. Cesta délky $m$ (s $m$ hranami) je graf
  $({0, \dots, m}, {{i, i+1} | i \in {0, \dots, m-1}})$.
\end{align*}

\card{Doplněk grafu $G$}
\begin{align*}
  Doplněk $\overline{G}$ grafu $G = \pair{G}{V}$ je graf $\pair{V}{\binom{V}{E} \setminus \E}$.
\end{align*}

\card{Izomorfismus grafů}
\begin{align*}
  Nechť $G$ a $H$ jsou dva grafy. Funkce $f: V(G) \rightarrow V(H)$, která:
  \begin{itemize}
    \item je bijekcí,
    \item pro každou dvojici $u, v \in V(G)$ platí: $\pair{u,v} \in E(G) \Leftrightarrow {f(u), f(v)} \in E(H)$.
  \end{itemize}
\end{align*}

\card{Automorfismus}
\begin{align*}
  Automorfismus G je izomorfismus se sebou samý, tedy $f: V(G) \rightarrow V(G)$, která:
  \begin{itemize}
    \item je bijekcí,
    \item pro každou dvojici $u, v \in V(G)$ platí: $\pair{u,v} \in E(G) \Leftrightarrow {f(u), f(v)} \in E(G)$.
  \end{itemize}
\end{align*}

\card{Stupeň vrcholu $deg_G(v)$}
\begin{align*}
  Počet hran grafu G obsahujících vrchol $v$.
\end{align*}

\card[Uzavřené okolí]{Okolí stupně $N_G(v)$}
\begin{align*}
  Množina všech sousedů vrcholu $v$ v grafu $G$.\\
  Množinu $N_G[v] = N_G(v) \cup \{v\}$ nazveme uzavřené okolí.
\end{align*}

\card{Regulární graf}
\begin{align*}
  Graf $G$ je $r$-regulární, pokud stupeň každého vrcholu je $r$.\\
  Graf je regulární, pokud je $r$-regulární pro nějaké $r$.
\end{align*}

\car{Princip sudosti a jeho důsledek}
\begin{align*}
  Pro každý graf $G = \pair{V}{E}$ platí\\
  $$\sum_{v \in V} deg_{G}(v) = 2|E|$$.\\
  Z tohoto vztahu plyne, že počet vrcholů lichécho stupně je sudý.
\end{align*}

\card[Matice sousednosti]{Reprezentace grafu}
\begin{align*}
  Čtvercová matice $A_{G} = (a_{ij})_{i,j}^n$ je definována předpisem:
  \[ \begin{cases}
        1 & pokud \{v_{i}, v_{j}\} \in E \\
        0 & jinak
     \end{cases}
  \]
\end{align*}

\card[Seznam sousedů]{Reprezentace grafu}
\begin{align*}
  Pro každý vrchol $v$ grafu $G$ uchováváme seznam sousedů (např. spojový seznam).
  Paměťová složitost je $|V| + 2|E|$.
\end{align*}

\card{Indukovaný podgraf}
\begin{align*}
  Graf $H$ je indukovaný podgraf grafu $G$, když\\
  $V(H) \subseteq V(G)$ a $E(H) = E(G) \cap \binom{V(H)}{2}$.\\
  Podgraf se značí $H \leq G$.
\end{align*}

\card{Sled}
\begin{align*}
  Doplnit
\end{align*}

\card{Cesta v grafu}
\begin{align*}
  Podgraf izomorfní nějaké cestě $P$. Délka cesty je počet hran.\\
  V ohodnoceném grafu je pak délka součtem ohodnocení jednotlivých hran.
\end{align*}

\card{Souvislá komponenta}
\begin{align*}
  Indukovaný podgraf $H$ grafu $G$, který:
  \begin{itemize}
    \item je souvislý ($\forall u, v \in V: \exists P\pair{u}{v}$),
    \item neexistuje podgraf $F$, $F \neq H$ takový že $H \subset F$.
  \end{itemize}
\end{align*}

\end{center}
\end{document}
