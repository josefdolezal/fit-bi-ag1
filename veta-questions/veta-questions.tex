\documentclass[20pt]{extarticle}
\usepackage[utf8]{inputenc}
\usepackage[czech]{babel}
\usepackage{t1enc}
\usepackage{amsfonts}
\usepackage{amsmath}
\usepackage{mathtools}
\usepackage{xcolor}


% iPhone 7 plus: 9:16
% showframe
\usepackage[paperheight=4.5in,paperwidth=8in,margin=0.5in,heightrounded]{geometry}

% increase math line height
\addtolength{\jot}{1em}

\pagenumbering{gobble}

\newcommand{\card}[3][]{
	\vspace*{\fill}

	\newpage
	\topskip0pt
	\vspace*{\fill}
		\Large #2

		\vspace{1cm}
		\normalsize #1
	\vspace*{\fill}
	\newpage

	\small \textcolor{gray}{#2 #1}
	\topskip0pt
	\vspace*{\fill}

	\normalsize
	#3
	\vspace*{\fill}
}
\newcommand{\splitline}{\noindent\rule{8cm}{0.4pt}}
\newcommand{\pair}[2]{\left(#1, #2\right)}
\newcommand{\graph}{G=\pair{V}{E}}
\newcommand{\edges}[1][G]{\E(#1)}
\newcommand{\vertexes}[1][G]{V(#1)}

\DeclarePairedDelimiter{\ceil}{\lceil}{\rceil}
\DeclarePairedDelimiter{\floor}{\lfloor}{\rfloor}

\begin{document}

\begin{center}

\topskip0pt
\vspace*{\fill}
\Large Josef Doležal\\[1cm]
\normalsize Veta otázky \textsc{bi-ag1}\\
\normalsize ZS 2016/17
\normalsize

\card[zdroj a stok]{Acyklický orientovaný graf}{
	\begin{description}
		\item[Acyklický orientovaný graf] Uspořádaná dvojice $\pair{V}{G}$, kde $V$
		je neprázdná množina vrcholů a $E$ množina orientovaných hran taková, že neobsahuje cyklus.
		\item[Zdroj] Vrchol, do kterého nevede žádná hrana.
		\item[Stok] Vrchol, ze kterého nevede žádná hrana.
	\end{description}
}

\card{Věta o existenci zdroje v orientovaném grafu}{
  Každý orientovaný graf, který neobsahuje cyklus, má alespoň jeden zdroj.
}

\card{Věta o existenci zdroje v orientovaném grafu}{
  Každý orientovaný graf, který neobsahuje cyklus, má alespoň jeden zdroj.
}

\card{Algoritmus topologického uspořádání orientovaného grafu}{
  \small
  \begin{enumerate}
    \item Zařaď do fronty všechny vrcholy se vstupním stupněm 0.
    \item (dokud není fronta prázdná) Vyber vrchol z počátku fronty
		\begin{itemize}
			\item - Vypiš vybraný vrchol
			\item - Pro každého následníka zkontrontroluj, jestli po odstranění hrany má vstupní stupeň 0.
			\item - Pokud má stupeň 0, přidej ho do fronty.
		\end{itemize}
  \end{enumerate}
	\normalsize
}

\card{Topologické uspořádání orientovaného grafu}{
  Topologické uspořádání orientovaného acyklického grafu $G = \pair{V}{E}$ je takové pořadí vrcholů $v_1, v_2, \dots, v_n$ grafu $G$,
  že pro každou hranu $\pair{v_i}{v_j} \in E$ platí $i < j$.
}

\card{Neorientovaný graf}{
  Neorientovaný graf je uspořádaná dvojice $\pair{V}{E}$, kde
  \begin{enumerate}
    \item $V$ je množina vrcholů
    \item $E$ je množina hran
  \end{enumerate}
  Hrana je dvouprvková podmnožina $V$.
}

\card{Orientovaný graf}{
  Orientovaný graf G je uspořádaná dvojice $\pair{V}{E}$, kde
  \begin{itemize}
    \item $V$ je neprázdná konečná množina vrcholů
    \item $E$ je množina orientovaných hran
  \end{itemize}
  Orientovaná hrana $\pair{u}{v} \in E$ je uspořádaná dvojice různých vrcholů $u, v \in V$.\\
	Říkáme, že $u$ je předchůdce $v$ a $v$ je následník $u$.
}

\card{Úplný graf $K_n$}{
  Úplný graf na $n$ ($n \ge 1 $) vrcholech $K_n$ je graf $\pair{v}{\binom{V}{2}}$, kde $|V| = n$.
}

\card{Úplný bipartitní graf $K_n$}{
  Nechť $n \geq 1$ a $m \geq 1$.
  Úplný bipartitní graf $K_{n,m}$ s $n$ vrcholy v jedné partitě a $m$ vrcholy v druhé
  partitě je graf $(A \cup B, \{\{a,b\} | a \in A,b \in B\})$, kde\\
  $A \cap B = \emptyset, |A| = n$ a $|B| = m$.
}

\card{Kružnice $C_n$}{
  Nechť $n \geq 1$. Kružnice délky $n$ (s $n$ vrcholy) je graf
  $({1, \dots, n}, {{i, i+1} | i \in {1, \dots, n-1}} \cup {{1, n}})$.
}

\card{Cesta $P_m$}{
  Nechť $m \geq 0$. Cesta délky $m$ (s $m$ hranami) je graf
  $({0, \dots, m}, {{i, i+1} | i \in {0, \dots, m-1}})$.
}

\card{Doplněk grafu $G$}{
  Doplněk $\overline{G}$ grafu $\graph$ je graf $\pair{V}{\binom{V}{E} \setminus E}$.
}

\card{Izomorfismus grafů}{
  Nechť $G$ a $H$ jsou dva grafy. Funkce $f: V(G) \rightarrow V(H)$, která:
  \begin{itemize}
    \item je bijekcí,
    \item pro každou dvojici $u, v \in V(G)$ platí: $\pair{u,v} \in E(G) \Leftrightarrow {f(u), f(v)} \in E(H)$.
  \end{itemize}
}

\card{Automorfismus}{
  Automorfismus G je izomorfismus se sebou samý, tedy $f: V(G) \rightarrow V(G)$, která:
  \begin{itemize}
    \item je bijekcí,
    \item pro každou dvojici $u, v \in V(G)$ platí: $\pair{u,v} \in E(G) \Leftrightarrow {f(u), f(v)} \in E(G)$.
  \end{itemize}
}

\card{Stupeň vrcholu $deg_G(v)$}{
  Počet hran grafu G obsahujících vrchol $v$.
}

\card[Uzavřené okolí]{Okolí stupně $N_G(v)$}{
  Množina všech sousedů vrcholu $v$ v grafu $G$.\\
  Množinu $N_G[v] = N_G(v) \cup \{v\}$ nazveme uzavřené okolí.
}

\card{Regulární graf}{
  Graf $G$ je $r$-regulární, pokud stupeň každého vrcholu je $r$.\\
  Graf je regulární, pokud je $r$-regulární pro nějaké $r$.
}

\card{Princip sudosti a jeho důsledek}{
  Pro každý graf $G = \pair{V}{E}$ platí\\
  $$\sum_{v \in V} deg_{G}(v) = 2|E|$$.\\
  Z tohoto vztahu plyne, že počet vrcholů lichécho stupně je sudý.
}

\card[Matice sousednosti]{Reprezentace grafu}{
  Čtvercová matice $A_{G} = (a_{ij})_{i,j}^n$ je definována předpisem:
  \[ \begin{cases}
        1 & \{v_{i}, v_{j}\} \in E \\
        0 & \text{jinak}
     \end{cases}
  \]
}

\card[Seznam sousedů]{Reprezentace grafu}{
  Pro každý vrchol $v$ grafu $G$ uchováváme seznam sousedů (např. spojový seznam).
  Paměťová složitost je $|V| + 2|E|$.
}

\card{Indukovaný podgraf}{
  Graf $H$ je indukovaný podgraf grafu $G$, když\\
  $V(H) \subseteq V(G)$ a $E(H) = E(G) \cap \binom{V(H)}{2}$.\\
  Podgraf se značí $H \leq G$.
}

\card{Sled}{
  Doplnit
}

\card{Cesta v grafu}{
  Podgraf izomorfní nějaké cestě $P$. Délka cesty je počet hran.\\
  V ohodnoceném grafu je pak délka součtem ohodnocení jednotlivých hran.
}

\card{Souvislá komponenta}{
  Indukovaný podgraf $H$ grafu $G$, který:
  \begin{itemize}
    \item je souvislý ($\forall u, v \in V: \exists P\pair{u}{v}$),
    \item vyvrací existenci souvislého podgrafu $F$, $F \neq H$ takového že $H \subseteq F$.
  \end{itemize}
	V inkluzi se jedná o maximální souvislý podgraf.
}

\card[Prohledávání do hloubky]{DFS}{
	Po výběru počátečního vrcholu $p$ z $V$ spuštěn rekurzivní algoritmus:
	\begin{itemize}
		\item Pokud je $p$ otevřený vrať se ($return$).
		\item Označ $p$ jako otevřený.
		\item Pro každého následníka spusť rekurzivně $DFS$ algoritmus.
		\item Po projití všech následníků označ uzel jako uzavřený.
	\end{itemize}
}

\card[$P_{m}$]{Orientovaná cesta}{
	Nechť $m \geq 0$. Orientovaná cesta s $m$ hranami $P_{m}$ je graf\\
	$(\{0, \dots, m\}, \{\pair{i}{i+1} | i \in \{0, \dots, m-1\})$.\\
	(Oproti standardní cestě se jedná o množinu uspořádaných dvojic)
}

\card[$C_{n}$]{Orientovaná kružnice}{
	Nechť $n \geq 2$. Orientovaná kružnice s $n$ vrcholy je graf\\
	$(\{1, \dots, n\}, \{\pair{i}{i+1} | i \in \{1, \dots, n-1\}\} \cup \{\pair{n}{1}\})$
}

\card[$deg_{G}^{+}(v)$]{Vstupní stupeň}{
	Počet orientovaných hran hran orientovaného grafu $G$\\
	končících ve vrcholu $v$.
}

\card[$deg_{G}^{+}(v)$]{Výstupní stupeň}{
	Počet orientovaných hran hran orientovaného grafu $G$\\
	vycházejících z vrcholu $v$.
}

\card[orientovaného grafu]{Symetrizace}{
	Neorientovaný graf $sym(G)= \pair{V'}{G'}$ kde $V' = V$, a\\
	${u, v} \in E'$ právě když $\pair{u}{v} \in E$ nebo $\pair{v}{u} \in E$.
}

\card{Slabá souvislost}{
	Graf $G=\pair{V}{E}$, jehož symetrizace $sym(G)$ je souvislá.
}

\card{Silná souvislost}{
	Graf, kde pro každé vrcholy $u, v \in V$ existuje orientovaná cesta z $u$ do $v$\\
	a současně existuje orientovaná cesta z $v$ do $u$ (ne nutně ta samá).
}

\card{Strom, les a list}{
	\begin{description}
		\item[Strom] Graf $G$, který je souvislý a acyklický.
		\item[Les] Graf $G$, který neobsahuje kružnice (nesouvislý, komponenty jsou stromy).
		\item[List] Vrchol $v$ jehož stupeň $deg_{G}(v)=1$.
	\end{description}
}

\card{Tvrzení o existenci listů}{
	Každý strom $T$ s alespoň 2 vrcholy obsahuje alespoň 2 listy.\\
	Lze dokázat pomocí hledání nejdelší cesty.
}

\card{Věta o trhání listů}{
	Je-li $\graph$ graf na alespoň 2 vrcholech a\\
	$v \in \vertexes$ je list. Pak:
	\begin{itemize}
		\item $G$ je strom.
		\item $G - v$ je strom.
	\end{itemize}
}

\card{Vlastnosti stromů}{
	\begin{itemize}
		\item $G$ je strom.
		\item Pro každé dva vrcholy $u,v \in V$ existuje právě jedna cesta z $u$ do $v$.
		\item $G$ je souvislý a vynecháním libovolné hrany vznikne nesouvislý graf.
		\item $G$ je souvislý a platí $|V|=|E|+1$.
	\end{itemize}
}

\card{Kostra grafu}{
	Nechť $G$ je souvislý.\\
	Podgraf $K$ grafu $G$ nazveme kostrou $G$, pokud\\
	$\vertexes[K]=\vertexes[G]$ a $K$ je strom.
}

\card{Vzdálenost dvou vrcholů $d\pair{u}{v}$}{
	Délka nejkratší cesty v $G$ spojující $u$ a $v$.\\ Pokud cesta neexistuje (jsou z jiných komponent),\\
	pak $d\pair{u}{v}=\infty$.
}

\card[BFS]{Prohledávání do šířky}{
    \small
	$DFS$ začíná výběrem počátečního vrcholu $s$ a dej mu hodnotu 0, následně:
	\begin{itemize}
		\item Označ všechny vrcholy jako nenalezené.
		\item Přidej $s$ do fronty.
		\item Dokud není fronta prázdná
		\begin{itemize}
			\item Odeber vrchol fronty $v$ a pro každého jeho následníka $w$:
			\item Je-li $w$ nenalezený, označ ho jako nalezený a jeho hodnotu nastav na hodnotu $v + 1$, $w$ přidej do fronty
		\end{itemize}
	\end{itemize}
	\normalsize
}

\card{Vlastnosti kostry BFS}{
	Doplnit
}

\card[BubbleSort]{Řadící algoritmy}{
	Funguje na principu probublávání velkých prvků. Algoritmus vezme dva prvky a pokud
	jsou ve špatném pořadí, prohodí je. Následně se posouvá o prvek dál.\\
	Ukončuje se ve chvíli, kdy v jednom běhu neproběhlo žádné prohození.\\
	Složitost $O(n^2)$, stabilní, in-place, datově citlivý.
}

\card[SelectSort]{Řadící algoritmy}{
	Funguje na principu vyhledávání nejnižšího prvku. Vstup se rozdělí na seřazenou
	a neseřazenou posloupnost. V každém kroku se vybere minimum z neseřazené a vloží
	se na konec seřazené. Volné místo se vyplní sešoupnutím prvků.\\
	Složitost $O(n^2)$, nestabilní, in-place a datově necitlivý.
}

\card[InsertSort]{Řadící algoritmy}{
	Na principu řazení vkládáním. Vstup se rozdělí na seřazenou a neseřazenou posloupnost.
	V každém kroku se vezme první prvek neseřazené posloupnosti a vloží se na správné
	místo v seřazené.\\
	Složitost $O(n^2)$ (v lepším případě $O(n)$), stabilní, in-place a datově citlivý.
}

\card[TopSort]{Třídící algoritmus}{
	\small
	\begin{itemize}
		\item Pro každou hranu $\pair{u}{v}$, proveď $D(u)+=1$. Všechny vrcholy $v$, které mají
		$D(v)=0$ přidej do fronty.
		\item Dokud není fronta prázdná vezmi a vypiš vrchol $v$ z čela fronty a pro každou
		hranu směřující z něho do $w$ proveď $D(w)-=1$, pokud nyní $D(w)=0$, zařaď ho do fronty.
	\end{itemize}
	\normalsize
}

\card[Vlastnosti]{Řadící algoritmy}{
	\begin{description}
		\item[Pamětová náročnost] Rozlišují se In-place a Out-of-place algoritmy.
		\item[Stabilita] Stabilní, pokud správně seřazené prvky ze vstupu mají stejné pořadí i na výstupu.
		\item[Citlivost] Určuje, jestli se mění časová složitost na základě vstupu.
	\end{description}
}

\card[předek, potomek, otec a syn]{Zakořeněný strom}{
	\begin{description}
		\item[Zakořeněný strom] Uspořádaná dvojice $\pair{T}{k}$, kde $k \in V(T)$
		je jeden zvolený vrchol stromu $T$ zvaný \textbf{kořen}.
		\item[Předek a potomek] Leží-li $u$ na cestě z $v$ do kořene, pak je
		$u$ \textbf{předek} a $v$ \textbf{potomek}.
		\item[Otec a syn] Pokud je navíc $\{u, v\} \in E(T)$ hrana, $u$
		je \textbf{otec} a \textbf{syn}.
	\end{description}
}

\card{Binární strom}{
	Strom, který splňuje:
	\begin{itemize}
		\item je zakořeněný,
		\item každý vrchol má nejvýše dva syny,
		\item u synů rozlišujeme, který je pravý a který levý.
	\end{itemize}
}

\card{Binární minimová halda}{
	Struktura tvaru binárního stromu, splňující:
	\begin{itemize}
		\item \textbf{Tvar haldy}: Strom má všechny hladiny kromě poslední plně obsazené.
		Poslední hladina je zaplně zleva doprava.
		\item \textbf{Haldové uspořádání}: Je-li $v$ vrchol a $s$ jeho syn,
		pak platí $k(v)<k(s)$.
	\end{itemize}
}

\card[binární haldy]{Počet hladin}{
	Binární halda s $n$ prvky má $\floor{\log n} + 1$  hladin.
}

\card[vložení prvku]{Binární halda}{
	Binární halda dovoluje vložit prvek na pozici listu. Tímto ale mohlo být
	porušeno haldové pravidlo. Je tedy potřeba prvek \textit{probublat} na
	správné místo. Probulání probíhá provnáním s hodnotou v rodiči (pokud je v
	rodiči větší, prohodí se).\\
	Složitost je $O(\log n)$.
}

\card[odstranění minima]{Binární halda}{
	Odstranit minimum není triviálně možné. Lze ho ale prohodit s nepravějším
	listem, následně odstranit a list probublat dolů na správné místo.\\
	Složitost je $O(\log n)$.
}

\card[reprezentace polem]{Binární halda}{
	Pro reprezentaci haldy lze snadno využít pole. Pokud uzly označíme čísly
	$1, \dots, n$, pak pro vrchol $v$ s indexem $i$ platí:
	\small
	\begin{itemize}
		\item pravý syn má index $2i + 1$,
		\item levý syn má index $2i$,
		\item otec má index $\floor{\frac{i}{2}}$,
		\item číslo $i \mod 2$ udává, zda-li $v$ je pravý syn
	\end{itemize}
	\normalsize
}

\card[algoritmus BuildHeap]{Binární halda}{
	Haldu lze složit v čase $O(n)$ zabubláním prvků, které nejsou listy.\\
	Algoritmus vezme vrcholy $\floor{\frac{n}{2}}, \dots, 1$ a postupně na ně zavolá
	operaci BubbleDown.
}

\card[řazení HeapSort]{Binární halda}{
	Prvky $x_1, \dots, x_n$ vložíme do pole a zavoláme na něj \textit{BuildHeap}.\\
	Nyní opakovaně voláme \textit{HeapExtractMin} a hodnoty ukládáme do výstupního pole.\\
	Složitost řazení je $O(n \log n)$.
}

\card[nafukovacího pole]{Amortizovaná analýza}{
	Uvažujme prázdné nafukovací pole. Potom celková časová složitost posloupnosti
	$n$ operací \textbf{NPInsert} je $O(n)$, neboli amortizovaná složitost je $O(1)$.
}

\card[binární sčítačky]{Amortizovaná analýza}{
	Uvažujme vynulovanou birnání sčítačku. Potom celková složitost posloupnosti
	$n$ volání operace \textit{Inc} měřená počtem bitových inverzí je $O(n)$,
	neboli amortizovaná složitost je $O(1)$.
}

\card[$B_k$]{Binomiální strom}{
	Uspořádaný (záleží na pořadí synů) zakořeněný strom, pro který platí:
	\begin{enumerate}
		\item $B_0$ je tvořen pouzen kořenem,
		\item Pro $k \geq 1$ získáme $B_k$ ze stromů $B_0, B_1, \dots, B_{k-1}$ tak,
		že přidáme nový kořen a kořeny těchto stromů (v tomto pořadí) přidáme jako jeho syny.
	\end{enumerate}
}

\card[Binomiálního stromu $B_k$]{Počet hladin, vrcholů a stupeň kořene}{
	Počet hladin $B_k$ je $k+1$, počet vrcholů je $2^k$ a stupeň kořene je $k$.
}

\card[Binomiálního stromu $B_k$]{Počet vrcholů v hloubce $i$}{
	Počet vrcholů stromu $B_k$ v hloubce $i$, $0 \leq i \leq k$, je\\
	$n_k(i)=\binom{k}{i}$.
}

\card{Binomiální minimová halda}{
	\small
	Halda obsahující $n$ prvků se skládá ze souboru binomiálních stromů
	$T=T_1, \dots, T_l$, kde:\\
	\begin{itemize}
		\item Pro každý strom $T_i$ platí haldové uspořádání.
		\item V souboru $T$ se žádný řád binomiálního stromu nevyskytuje vícekrát.
		\item Soubor stromů je uspořádán vzestupně.
	\end{itemize}
	\normalsize
}

\card[v Binomiální haldě]{Výskyt stromu $T$}{
	Binomiální strom $B_k$ se v souboru stromů $n$-prvkové binomiální haldy vyskytuje
	právě tehdy, když je ve dvojkovém zápisu čísla $n$ $k$-tý nejnižší bit
	nastaven na 1.
}

\card[Binomiálních hald]{Sloučení}{
  \footnotesize
	Sloučení hald probíhá obdobně jako sčítání binárních čísel pod sebou.
	Postupně se slučují stromy stejného řádu způsobem:
	\begin{itemize}
		\item Pokud obě haldy obsahují stromy slučovaného řádu, vznikne strom $B_{k+1}$
		a bere se jako přenos.
		\item Pokud mám přenos a obě haldy mají stromy se stejným řádem, vypíše se
		přenos a sločí se stromy $B_{k+1}$ čímž vzniká nový přenos.
		\item Pokud mám přenos a jen jedna halda obsahuje strom $B_{k+1}$, sloučím
		ho s přenosem a vznikne nový přenos.
		\item Pokud mám přenos a ani jedna halda neobsahuje strom $B_{k+1}$, vypíšu ho.
	\end{itemize}
	Časová složitost je $O(\log n)$.
	\normalsize
}

\card[Binomiálních hald]{Vložení prvku}{
	Probíhá pomocí sloučení hald (z prvku se stane $B_0$).\\
	Časová složitost $\Theta(1)$.
}

\card[Binomiálních hald]{Odebrání minima}{
	\small
	\begin{itemize}
		\item Najdeme v haldě strom $T$, jehož kořen je minimum. Odpojíme $T$ z haldy.
		\item Odtrhneme všechny syny a vložíme je do nové haldy $H'$.
		\item Sloučíme BHMerge$(H, H')$.
	\end{itemize}
	Časová složitost je $O(\log n)$.
	\normalsize
}

\card[BVS]{Binární vyhledávací strom}{
	Binární strom, v jehož každém vrcholu $v$ je uložen unikátní klíč $k(v)$.
	Pro každý $v$ pak musí platit:
	\begin{itemize}
		\item Pokud $a$ je v levém podstromu, pak $k(a) < k(v)$.
		\item Pokud $b$ je v levém podstromu, pak $k(b) > k(v)$.
	\end{itemize}
}

\card[Vložení prvku]{Binární vyhledávací strom}{
	\small
	Vkládám vrchol $k$ do stromu $T$:
	\begin{itemize}
		\item Pokud $T$ je prázdný, vytvoř vrchol z $k$ a vrať ho.
		\item (nebo) Pokud $k$ je menší než kořen, vlož $k$ do $L(T)$.
		\item (nebo) Pokud $k$ je větší než kořen, vlož $k$ do $R(T)$.
	\end{itemize}
	Časová složitost je $O(n)$.
}

\card[Odstranění prvku]{Binární vyhledávací strom}{
	\small
	\begin{itemize}
		\item (nebo) Pokud je $k$ je menší než kořen, mažeme z levého podstromu.
		\item (nebo) Pokud je $k$ je větší než kořen, mažeme z pravého podstromu.
		\item (nebo) Pokud se rovná a má jednoho syna, nahradíme ho synem.
		\item (nebo) Pokud se se rovná a má oba syny, nahradíme ho následníkem.
	\end{itemize}
	Časová složitost je $O(n)$.
}

\card[nalezení následníka]{Binární vyhledávací strom}{
	\small
	Při hledání následníka vrcholu $v$ může nastat:
	\begin{itemize}
		\item $v$ má pravého syna: následník je minimum z pravého podstromu
		\item $v$ je levý syn, vrať otce $v$
		\item (jinak) $u$ je otec $v$, dokud $u$ je pravým synem: $u \leftarrow$ otec($u$), následně vrať minimum z otce $u$ (může být $null$)
	\end{itemize}
}

\card[nalezení předchůdce]{Binární vyhledávací strom}{
	\small
	Při hledání předchůdce vrcholu $v$ může nastat:
	\begin{itemize}
		\item $v$ má levého syna: následník je maximum z pravého podstromu
		\item $v$ je pravý syn, vrať otce $v$
		\item (jinak) $u$ je otec $v$, dokud $u$ je levým synem: $u \leftarrow$ otec($u$), následně vrať maximum z otce $u$ (může být $null$)
	\end{itemize}
}

\card{AVL Strom}{
	Binární vyhledávací strom, kde pro každý vrchol $v$ platí, že\\
	$|h(l(v)) - h(r(v)) \leq 1|$.\\
	Tedy, že hloubky podstromů se mohou lišit maximálně o 1.
}

\card[Jednoduchá rotace]{AVL Strom}{
	Pokud se ve vrcholu $v$ liší hloubky pod stromů více neže o 1 a zároveň
	do $v$ se nerovnost propagovala z levého (L), resp. pravého (R) syna a do toho
	z jeho levého (L), resp. pravého (R) syna pak provedeme LL, resp. RR rotaci.
}

\card[Dvojitá rotace]{AVL Strom}{
	Pokud se ve vrcholu $v$ liší hloubky pod stromů více neže o 1 a zároveň
	do $v$ se nerovnost propagovala z levého (L), resp. pravého (R) syna a do toho
	z jeho levého (R), resp. pravého (L) syna pak provedeme LR, resp. RL rotaci.
	Jednoduchou rotací lze přejít do stavu LL nebo RR.
}

\card{Hashování}{
	Pro univerzum klíčů $U$ zvolíme konečnou množinu přihrádek $P={0, \dots, m}$
	(\textbf{hashovací tabulku}) a \textbf{hashovací funkci} $h: U \to P$, která
	každému klíči přiřadí jednu přihrádku. Množinu klíčů $K \subset U$ umisťujeme
	do přihrádek $\forall k \in K: h(k)$.
}

\card[s řetízky]{Hashování}{
	Hashovací tabulka je pole $m$, přihrádek, které jsou buď prázdné nebo v nich
	začínají spojové seznamy uložených prvků.
}

\card[s otevřenou adresací]{Hashování}{
	Každý klíč $k \in U$ má \textbf{vyhledávací posloupnost} $h(k, 0), \dots, h(k, m-1)$,
	která určuje v jakém pořadí se budou přihrádky zkoušet. Pokud algoritmus narazí
	na zaplněnou přihrádku, zkusí další z posloupnosti.
}

\card[s lineárním přidáváním]{Hashování}{
	Prohledávací posloupnost je dána funkcí\\
	$h(k, i) = (f(k) + i) \mod m$\\, kde
	$f(k)$ je hashovací funkce a $i$ je počet neúspěšných pokusů. Funkce tedy zkouší
	přihrádky jdoucí po sobě.
}

\card[s dvojitým hashováním]{Hashování}{
	Hashovací posloupnost je dána funkcí $h(k, i) = (f(k) + i \cdot g(k) \mod m)$
	kde $f$ a $g$ jsou různé hashovací funkce a $i$ je počet neúspěšných pokusů.
}

\card{MergeSort}{
	Vsupní posloupnost $n$ prvků rozdělíme na $\floor{\frac{n}{2}}$ a $\ceil{\frac{n}{2}}$
	prvků a na obě poloviny zavoláme rekurzivně stejný algoritmus. Rekurze se zastavuje
	u jednoprvkové posloupnosti, která je seřazená, Při návratu z rekurze se seřazené
	posloupnosti slévají do jedné. Časová i paměťová složitost $\Theta(n)$.
}

\card[$x$ a $y$]{Rychlé násobení čísel}{
	Spočívá v rozdělení čísel $x$ a $y$ na $x=a10^{\frac{n}{2}} + b$, $x=c10^{\frac{n}{2}} + d$.\\
	V prvním kroku lze vyjádřit: $xy = ac10^n + (ad + bc)10^{\frac{n}{2}} + bd$ s $\Theta(n)$.\\
	V druhém kroku lze rozepsat $(ad + bc) = ((a+b)(c+d) - ac - bd)$ a dospět k\\
	$xy = ac10^n + ((a+b)(c+d) - ac - bd)10^{\frac{n}{2}} + bd$\\ se složitostí $\Theta(n^{\log 3})$.
}

\card[k-tý nejmenší prvek]{QuickSelect}{
	\small
	Rekurzivní algoritmus, zvolím si pivota $p$ a posloupnost rozdělím na $L$ (prvky menší
	než $p$), $P$ (prvky větší než $p$) a $S$ (prvky rovné $p$).
	\begin{itemize}
		\item Pokud $k \leq |L|$ vrať $QuickSelect(L, k)$
		\item (nebo) je-li $k \leq |L|+|S|$ vrať $p$
		\item (nebo) vrať $QuickSelect(P, k-|L|-|S|)$
	\end{itemize}
}

\card{QuickSort}{
	Obdobně jako \textit{QuickSelect} rozděluje posloupnost na $L$, $P$ a $S$ podle pivota $p$.
	Rekurzivně seřadí $L$ a $P$ a části spojí za sebe a vrátí.
}

\card[v porovnávacím modelu]{Dolní mez složitosti vyhledávání}{
	Každý deterministický algoritmus v porovnávacím modelu, který nalezne prvek v
	$n$-prvkové seřazené posloupnsti, pužije nejméně $O(\log n)$ porovnání.
}

\card[v porovnávacím modelu]{Dolní mez složitosti řazení}{
	Každý deterministický algoritmus v porovnávacím modelu, který seřadí $n$-prvkovou
	posloupnost použije alespoň $O(n \log n)$ porovnání.
}

\card{CountingSort}{
	Řadí $n$ prvků z množiny $\{0, \dots, r\}$. Nejprve projde pole a spočítá,
	kolikrát tam který prvek je. Z tohoto počtu vypočte následně na jakou pozici
	prvek ve výstupním poli umístí (drží si tabulku kde si napočítá indexy). Když
	pak prochází vstupní pole, dívá se do tabulky indexů a index inkrementuje.
}

\card{LexCountingSort}{
	Používá se k řazení $n$ $k$-tic z množiny $\{1, \dots, r\}^k$.\\
	Algoritmus řadí pomocí algoritmu \textit{CountingSort} podle $i$-té souřadnice
	zleva (vezme $i$-tou souřadnici všech $k$-tic a podle ní odzadu řadí).\\
	Paměťová složitost $\Theta(k \cdot (n+r))$, časová složitost $\Theta(kn+r)$.
}

\card{Nejdelší rostoucí podposloupnost}{
	Pole se prochází odzadu. Pro každý prvek se udržuje délka nejdelší rostoucí posloupnosti.
	Pro každý prvek $v$ se nejprve uloží hodnota 1 a následně se prochází všechny prvky $w$ za ním.
	Pokud je prvek $h(v) < h(w) + 1$ pak $h(v) = h(w) + 1$. Na konci pro každé číslo
	máme nalezenu nejdelší rostoucí podposloupnost.
}

\card{Triangulace kovexního mnohoúhelníku}{
	\small
	Rozdělení mnohoúhelníku na trojúhelníky pomocí diagonál (hrana $a_i a_j$ kde
	$a_i$ a $a_j$) jsou nesousední vrcholy. Každá triangulace má $n-3$ diagonál.
	Minimální triangulace je nejmenší ze součtů délek diagonál.\\
	Pro každý vrchol $j = \{1, \dots, n\}$ číslo $M[i, j]$ vyjadřuje MTM $i$ vrcholů
	za sebou počínaje $j$. Definujeme $M[2,j] = 0$ a $M[3, j] = |a_ja_{j+2}|$\\
	Následně pak platí: $M[i,j] = min_{1\leq k\leq i-2}(M[k+1, j] + M[i-k, j+k]) + ||a_ja_{j+i-1}||$.
	MTM je $M[n, 1] - ||a_1a_n||$.
}

\card{Editační vzdálenost řetězců}{
	\textbf{Editační operace} na řetězci je \textit{vložení}, \textit{smazání} nebo
	\textit{změna} jednoho znaku.
	\\\textbf{Editační vzdálenost} řetězců $x=x_1, \dots, x_n$
	a řetězců $y=y_1, \dots, y_m$ zn. $L(x, y)$ je nejmenší počet editačních operací
	potřebných k tomu, abychom z prvního řetězce vytvořili druhý.
}

\card[algoritmus]{Editační vzdálenost řetězců}{
	\small
	Rekurzivní algoritmus vzdálenosti suffixů (tedy měří se od konce). V každé iteraci
	následně provedu $EditRec(i,j)$:
	\begin{itemize}
		\item Pokud $i > n$ vracím $m-j+1$, pokud $j > m$ vracím $n-i+1$
		\item $\ell_{z} = EditRec(i+1, j+1)$
		\item Pokud $x_i \neq y_j$: $\ell_z = \ell_z + 1$.
		\item $\ell_s = EditRec(i+1, j)$, $\ell_v = EditRec(i, j+1)$.
		\item Vrať $min\{\ell_z, \ell_s, \ell_v\}$.
	\end{itemize}
	\normalsize
}

\card[ohodnoceného grafu]{Minimální kostra}{
	Existuje $\graph$ souvislý neorientovaný graf a $w: E \to \mathbb{R}$ je \textbf{váhová funkce},
	která přiřazuje hranám čísla - váhy.\\
	Minimální kostra je taková, která má mezi všemi kostrami minimální váhu.
}

\card[hledání minimální kostry]{Jarníkův algoritmus}{
	\small
	\begin{itemize}
		\item Začíná se se stromem, který jeden vrchol a žádnou hranu.
		\item V dalším kroku se vybere incidentní hrana s tímto vrcholem s nejmenší váhou.
		\item Opakujeme postup: vybereme nejlehčí z hran, která vede ze stromu do původního grafu.
		Tento postup se opakuje dokud v původním grafu jsou vrcholy.
	\end{itemize}
	Časová složitost $O(nm)$, paměťová $O(n+m)$, kde $n=|V|$ a $m=|E|$.
	\normalsize
}

\card{Elementární řez grafu}{
	Nechť $A$ je podmožina vrcholů grafu $\graph$ a $B$ její doplněk ($B=V\setminus$A).
	Množina hran $\{a_i, a_j\}$ kde $a_i \in A$ a $a_j \in B$ je \textbf{elementární řez}
	grafu $G$ určený množinami $A$ a $B$.
}

\card{Lemma řezech grafu}{
	Nechť $G$ je graf s unikátními vahami, $R$ nějaký jeho elementární řez a $e$ nejlehčí
	hrana řezu tohoto řezu.\\
	Pak $e$ leží v každé minimální kostře grafu $G$.
}

\card[hledání minimální kostry]{Kruskalův algoritmus}{
	Nechť $G$ je graf s unikátními vahami, $R$ nějaký jeho elementární řez a $e$ nejlehčí
	hrana řezu tohoto řezu.\\
	Pak $e$ leží v každé minimální kostře grafu $G$.
}

\card[hledání minimální kostry]{Kruskalův algoritmus}{
	\small
	Na principu vyhledávání nejlevnějších hran (vynechávají se ty, které tvoří cyklus).
	\begin{itemize}
		\item Seřaď hrany podle váhy a vytvoř strom $T=(V, \emptyset)$.
		\item Pro $i=1,\dots,m$ opakuj: pokud krajní body hrany $e_i$ leží v různých
		komponentách, přidej je do stromu (v opačném případě tvoří cyklus)
	\end{itemize}
	Složitost je $O(m \log n + n^2)$.
}

\card[pomocí keříků]{Struktura Union-Find}{
	\small
	Struktura je reprezentovaná polem, kde si každý syn drží index rodiče. Find (
	udávající zda jsou dva vrcholy ve stejné komponentě) traverzuje přes rodiče
	až do kořene keříku a porovnává jestli mají stejný kořen. Každý kořen si drží
	svou hloubku. Při slučování dvou keřů se mělčí dát pod hlubší, tím se nezmění hlouba.
	Pokud mají stejnou hloubku, jeden se zapojí pod druhý a zvýší se hloubka nového kořene.\\
	Kruskalův algoritmus má následně složitost $O(m \log n)$.
}

\card[nejkratší cesta v ohodnoceném grafu]{Dijkstrův algoritmus}{
	\footnotesize
	\begin{itemize}
		\item Všechny označím jako nenalezené, $h(v) = \infty$, $h(v_0) = 0$, $v_0$ otevřený
		\item Dokud existuje otevřený vrchol $v$, vyber takový, který má nejmenší $h(v)$.
		Pro následníky $w$ vrcholu $v$:
		\begin{itemize}
			\item Pokud $h(w) > h(v) + l(v, w)$: $h(w) = h(v) + l(v, w)$, stav($w$) otevřený
			a $P(w) = v$.
			\item Po průchodu všemi sousedy zavři $v$.
		\end{itemize}
	\end{itemize}
	S použitím haldy je složitost $O((n+m) \cdot \log n).$\\
	Funguje i na grafech se záporným ohodnocením hran, ale bez záporných cyklů.
}

\card{Relaxace}{
	Relaxace je obecný případ Dijkstrova algoritmu, kde při vybírání otevřeného
	vrcholu se bere libovolný (nikoliv nejmenší). Na rozdíl od Dijkstry může
	jeden vrchol otevřít i uzavřít vícekrát.
}

\card[nejkratší cesta v ohodnoceném grafu]{Bellman-Fordův algoritmus}{
    \footnotesize
	Funguje na podobném principu jako Dijkstrův algoritmus. Místo vybírání
	prvku s nejnižším bere vrchol z čela fronty pro jeho následníky provádí:
	\begin{itemize}
		\item Pokud $h(w) > h(v) + l(v, w)$:
		\begin{itemize}
			\item $h(w) = h(v) + l(v, w)$
			\item Pokud stav($w$) $\neq$ otevřený, přidej $w$ do fronty
			\item stav($w$) nastav na otevřený
			\item uzavři $v$
		\end{itemize}
	\end{itemize}
	Oproti Dijkstrovi také znovu otevírá již uzavřené uzly.
	\normalsize
}

\card[nejkratší cesta v ohodnoceném grafu]{Bellman-Fordův algoritmus}{
    \footnotesize
	Funguje na podobném principu jako Dijkstrův algoritmus. Místo vybírání
	prvku s nejnižším bere vrchol z čela fronty pro jeho následníky provádí:
	\begin{itemize}
		\item Pokud $h(w) > h(v) + l(v, w)$:
		\begin{itemize}
			\item $h(w) = h(v) + l(v, w)$
			\item Pokud stav($w$) $\neq$ otevřený, přidej $w$ do fronty
			\item stav($w$) nastav na otevřený
			\item uzavři $v$
		\end{itemize}
	\end{itemize}
	Oproti Dijkstrovi také znovu otevírá již uzavřené uzly. Složitost $O(nm)$
	\normalsize
}

\card[nejkratší cesta v ohodnoceném grafu]{Zjednodušený Bellman-Fordův algoritmus}{
	V cyklu pro $i = 1, \dots, n$ vezmeme každou hranu a podíváme, jestli
	její cílový vrchol nemá větší hodnotu než počáteční zvývšený o váhu cesty, pokud ano
	vložíme novou hodnotu do $v$ a nastavíme mu jako předka vrchol $u$.
}

\end{center}
\end{document}
