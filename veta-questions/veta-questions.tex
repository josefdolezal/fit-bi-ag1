\documentclass[20pt]{extarticle}
\usepackage[utf8]{inputenc}
\usepackage[czech]{babel}
\usepackage{t1enc}
\usepackage{amsfonts}
\usepackage{amsmath}
\usepackage{xcolor}

% iPhone 7 plus: 9:16
% showframe
\usepackage[paperheight=4.5in,paperwidth=8in,margin=0.5in,heightrounded]{geometry}

% increase math line height
\addtolength{\jot}{1em}

\pagenumbering{gobble}

\newcommand{\card}[3][]{
	\vspace*{\fill}

	\newpage
	\topskip0pt
	\vspace*{\fill}
		\Large #2

		\vspace{1cm}
		\normalsize #1
	\vspace*{\fill}
	\newpage

	\small \textcolor{gray}{#2 #1}
	\topskip0pt
	\vspace*{\fill}

	\normalsize
	#3
	\vspace*{\fill}
}
\newcommand{\splitline}{\noindent\rule{8cm}{0.4pt}}
\newcommand{\pair}[2]{\left(#1, #2\right)}
\newcommand{\graph}{G=\pair{V}{E}}
\newcommand{\edges}[1][G]{\E(#1)}
\newcommand{\vertexes}[1][G]{V(#1)}

\begin{document}
\begin{center}

\topskip0pt
\vspace*{\fill}
\Large Josef Doležal\\[1cm]
\normalsize Veta otázky \textsc{bi-ag1}\\
\normalsize ZS 2016/17
\normalsize

\card[zdroj a stok]{Acyklický orientovaný graf}{
	\begin{description}
		\item[Acyklický orientovaný graf] Uspořádaná dvojice $\pair{V}{G}$, kde $V$
		je neprázdná množina vrcholů a $E$ množina orientovaných hran taková, že neobsahuje cyklus.
		\item[Zdroj] Vrchol, do kterého nevede žádná hrana.
		\item[Stok] Vrchol, ze kterého nevede žádná hrana.
	\end{description}
}

\card{Věta o existenci zdroje v orientovaném grafu}{
  Každý orientovaný graf, který neobsahuje cyklus, má alespoň jeden zdroj.
}

\card{Věta o existenci zdroje v orientovaném grafu}{
  Každý orientovaný graf, který neobsahuje cyklus, má alespoň jeden zdroj.
}

\card{Algoritmus topologického uspořádání orientovaného grafu}{
  \small
  \begin{enumerate}
    \item Zařaď do fronty všechny vrcholy se vstupním stupněm 0.
    \item (dokud není fronta prázdná) Vyber vrchol z počátku fronty
		\begin{itemize}
			\item - Vypiš vybraný vrchol
			\item - Pro každého následníka zkontrontroluj, jestli po odstranění hrany má vstupní stupeň 0.
			\item - Pokud má stupeň 0, přidej ho do fronty.
		\end{itemize}
  \end{enumerate}
	\normalsize
}

\card{Topologické uspořádání orientovaného grafu}{
  Topologické uspořádání orientovaného acyklického grafu $G = \pair{V}{E}$ je takové pořadí vrcholů $v_1, v_2, \dots, v_n$ grafu $G$,
  že pro každou hranu $\pair{v_i}{v_j} \in E$ platí $i < j$.
}

\card{Neorientovaný graf}{
  Neorientovaný graf je uspořádaná dvojice $\pair{V}{E}$, kde
  \begin{enumerate}
    \item $V$ je množina vrcholů
    \item $E$ je množina hran
  \end{enumerate}
  Hrana je dvouprvková podmnožina $V$.
}

\card{Orientovaný graf}{
  Orientovaný graf G je uspořádaná dvojice $\pair{V}{E}$, kde
  \begin{itemize}
    \item $V$ je neprázdná konečná množina vrcholů
    \item $E$ je množina orientovaných hran
  \end{itemize}
  Orientovaná hrana $\pair{u}{v} \in E$ je uspořádaná dvojice různých vrcholů $u, v \in V$.\\
	Říkáme, že $u$ je předchůdce $v$ a $v$ je následník $u$.
}

\card{Úplný graf $K_n$}{
  Úplný graf na $n$ ($n \ge 1 $) vrcholech $K_n$ je graf $\pair{v}{\binom{V}{2}}$, kde $|V| = n$.
}

\card{Úplný bipartitní graf $K_n$}{
  Nechť $n \geq 1$ a $m \geq 1$.
  Úplný bipartitní graf $K_{n,m}$ s $n$ vrcholy v jedné partitě a $m$ vrcholy v druhé
  partitě je graf $(A \cup B, \{\{a,b\} | a \in A,b \in B\})$, kde\\
  $A \cap B = \emptyset, |A| = n$ a $|B| = m$.
}

\card{Kružnice $C_n$}{
  Nechť $n \geq 1$. Kružnice délky $n$ (s $n$ vrcholy) je graf
  $({1, \dots, n}, {{i, i+1} | i \in {1, \dots, n-1}} \cup {{1, n}})$.
}

\card{Cesta $P_m$}{
  Nechť $m \geq 0$. Cesta délky $m$ (s $m$ hranami) je graf
  $({0, \dots, m}, {{i, i+1} | i \in {0, \dots, m-1}})$.
}

\card{Doplněk grafu $G$}{
  Doplněk $\overline{G}$ grafu $\graph$ je graf $\pair{V}{\binom{V}{E} \setminus E}$.
}

\card{Izomorfismus grafů}{
  Nechť $G$ a $H$ jsou dva grafy. Funkce $f: V(G) \rightarrow V(H)$, která:
  \begin{itemize}
    \item je bijekcí,
    \item pro každou dvojici $u, v \in V(G)$ platí: $\pair{u,v} \in E(G) \Leftrightarrow {f(u), f(v)} \in E(H)$.
  \end{itemize}
}

\card{Automorfismus}{
  Automorfismus G je izomorfismus se sebou samý, tedy $f: V(G) \rightarrow V(G)$, která:
  \begin{itemize}
    \item je bijekcí,
    \item pro každou dvojici $u, v \in V(G)$ platí: $\pair{u,v} \in E(G) \Leftrightarrow {f(u), f(v)} \in E(G)$.
  \end{itemize}
}

\card{Stupeň vrcholu $deg_G(v)$}{
  Počet hran grafu G obsahujících vrchol $v$.
}

\card[Uzavřené okolí]{Okolí stupně $N_G(v)$}{
  Množina všech sousedů vrcholu $v$ v grafu $G$.\\
  Množinu $N_G[v] = N_G(v) \cup \{v\}$ nazveme uzavřené okolí.
}

\card{Regulární graf}{
  Graf $G$ je $r$-regulární, pokud stupeň každého vrcholu je $r$.\\
  Graf je regulární, pokud je $r$-regulární pro nějaké $r$.
}

\card{Princip sudosti a jeho důsledek}{
  Pro každý graf $G = \pair{V}{E}$ platí\\
  $$\sum_{v \in V} deg_{G}(v) = 2|E|$$.\\
  Z tohoto vztahu plyne, že počet vrcholů lichécho stupně je sudý.
}

\card[Matice sousednosti]{Reprezentace grafu}{
  Čtvercová matice $A_{G} = (a_{ij})_{i,j}^n$ je definována předpisem:
  \[ \begin{cases}
        1 & \{v_{i}, v_{j}\} \in E \\
        0 & \text{jinak}
     \end{cases}
  \]
}

\card[Seznam sousedů]{Reprezentace grafu}{
  Pro každý vrchol $v$ grafu $G$ uchováváme seznam sousedů (např. spojový seznam).
  Paměťová složitost je $|V| + 2|E|$.
}

\card{Indukovaný podgraf}{
  Graf $H$ je indukovaný podgraf grafu $G$, když\\
  $V(H) \subseteq V(G)$ a $E(H) = E(G) \cap \binom{V(H)}{2}$.\\
  Podgraf se značí $H \leq G$.
}

\card{Sled}{
  Doplnit
}

\card{Cesta v grafu}{
  Podgraf izomorfní nějaké cestě $P$. Délka cesty je počet hran.\\
  V ohodnoceném grafu je pak délka součtem ohodnocení jednotlivých hran.
}

\card{Souvislá komponenta}{
  Indukovaný podgraf $H$ grafu $G$, který:
  \begin{itemize}
    \item je souvislý ($\forall u, v \in V: \exists P\pair{u}{v}$),
    \item vyvrací existenci souvislého podgrafu $F$, $F \neq H$ takového že $H \subseteq F$.
  \end{itemize}
	V inkluzi se jedná o maximální souvislý podgraf.
}

\card[Prohledávání do hloubky]{DFS}{
	Po výběru počátečního vrcholu $p$ z $V$ spuštěn rekurzivní algoritmus:
	\begin{itemize}
		\item Pokud je $p$ otevřený vrať se ($return$).
		\item Označ $p$ jako otevřený.
		\item Pro každého následníka spusť rekurzivně $DFS$ algoritmus.
		\item Po projití všech následníků označ uzel jako uzavřený.
	\end{itemize}
}

\card[$P_{m}$]{Orientovaná cesta}{
	Nechť $m \geq 0$. Orientovaná cesta s $m$ hranami $P_{m}$ je graf\\
	$(\{0, \dots, m\}, \{\pair{i}{i+1} | i \in \{0, \dots, m-1\})$.\\
	(Oproti standardní cestě se jedná o množinu uspořádaných dvojic)
}

\card[$C_{n}$]{Orientovaná kružnice}{
	Nechť $n \geq 2$. Orientovaná kružnice s $n$ vrcholy je graf\\
	$(\{1, \dots, n\}, \{\pair{i}{i+1} | i \in \{1, \dots, n-1\}\} \cup \{\pair{n}{1}\})$
}

\card[$deg_{G}^{+}(v)$]{Vstupní stupeň}{
	Počet orientovaných hran hran orientovaného grafu $G$\\
	končících ve vrcholu $v$.
}

\card[$deg_{G}^{+}(v)$]{Výstupní stupeň}{
	Počet orientovaných hran hran orientovaného grafu $G$\\
	vycházejících z vrcholu $v$.
}

\card[orientovaného grafu]{Symetrizace}{
	Neorientovaný graf $sym(G)= \pair{V'}{G'}$ kde $V' = V$, a\\
	${u, v} \in E'$ právě když $\pair{u}{v} \in E$ nebo $\pair{v}{u} \in E$.
}

\card{Slabá souvislost}{
	Graf $G=\pair{V}{E}$, jehož symetrizace $sym(G)$ je souvislá.
}

\card{Silná souvislost}{
	Graf, kde pro každé vrcholy $u, v \in V$ existuje orientovaná cesta z $u$ do $v$\\
	a současně existuje orientovaná cesta z $v$ do $u$ (ne nutně ta samá).
}

\card{Strom, les a list}{
	\begin{description}
		\item[Strom] Graf $G$, který je souvislý a acyklický.
		\item[Les] Graf $G$, který neobsahuje kružnice (nesouvislý, komponenty jsou stromy).
		\item[List] Vrchol $v$ jehož stupeň $deg_{G}(v)=1$.
	\end{description}
}

\card{Tvrzení o existenci listů}{
	Každý strom $T$ s alespoň 2 vrcholy obsahuje alespoň 2 listy.\\
	Lze dokázat pomocí hledání nejdelší cesty.
}

\card{Věta o trhání listů}{
	Je-li $\graph$ graf na alespoň 2 vrcholech a\\
	$v \in \vertexes$ je list. Pak:
	\begin{itemize}
		\item $G$ je strom.
		\item $G - v$ je strom.
	\end{itemize}
}

\card{Vlastnosti stromů}{
	\begin{itemize}
		\item $G$ je strom.
		\item Pro každé dva vrcholy $u,v \in V$ existuje právě jedna cesta z $u$ do $v$.
		\item $G$ je souvislý a vynecháním libovolné hrany vznikne nesouvislý graf.
		\item $G$ je souvislý a platí $|V|=|E|+1$.
	\end{itemize}
}

\card{Kostra grafu}{
	Nechť $G$ je souvislý.\\
	Podgraf $K$ grafu $G$ nazveme kostrou $G$, pokud\\
	$\vertexes[K]=\vertexes[G]$ a $K$ je strom.
}

\card{Vzdálenost dvou vrcholů $d\pair{u}{v}$}{
	Délka nejkratší cesty v $G$ spojující $u$ a $v$.\\ Pokud cesta neexistuje (jsou z jiných komponent),\\
	pak $d\pair{u}{v}=\infty$.
}

\card[BFS]{Prohledávání do šířky}{
    \small
	$DFS$ začíná výběrem počátečního vrcholu $s$ a dej mu hodnotu 0, následně:
	\begin{itemize}
		\item Označ všechny vrcholy jako nenalezené.
		\item Přidej $s$ do fronty.
		\item Dokud není fronta prázdná
		\begin{itemize}
			\item Odeber vrchol fronty $v$ a pro každého jeho následníka $w$:
			\item Je-li $w$ nenalezený, označ ho jako nalezený a jeho hodnotu nastav na hodnotu $v + 1$, $w$ přidej do fronty
		\end{itemize}
	\end{itemize}
	\normalsize
}

\card{Vlastnosti kostry BFS}{
	Doplnit
}

\card[BubbleSort]{Řadící algoritmy}{
	Funguje na principu probublávání velkých prvků. Algoritmus vezme dva prvky a pokud
	jsou ve špatném pořadí, prohodí je. Následně se posouvá o prvek dál.\\
	Ukončuje se ve chvíli, kdy v jednom běhu neproběhlo žádné prohození.\\
	Složitost \Omicron(n^2), stabilní, in-place, datově citlivý.
}

\card[SelectSort]{Řadící algoritmy}{
	Funguje na principu vyhledávání nejnižšího prvku. Vstup se rozdělí na seřazenou
	a neseřazenou posloupnost. V každém kroku se vybere minimum z neseřazené a vloží
	se na konec seřazené. Volné místo se vyplní sešoupnutím prvků.\\
	Složitost \Omicron(n^2), nestabilní, in-place a datově necitlivý.
}

\card[InsertSort]{Řadící algoritmy}{
	Na principu řazení vkládáním. Vstup se rozdělí na seřazenou a neseřazenou posloupnost.
	V každém kroku se vezme první prvek neseřazené posloupnosti a vloží se na správné
	místo v seřazené.\\
	Složitost \Omicron(n^2) (v lepším případě \Omicron(n)), stabilní, in-place a datově citlivý.
}

\card[TopSort]{Třídící algoritmus}{
	\small
	\begin{itemize}
		\item Pro každou hranu $\pair{u}{v}$, proveď $D(u)+=1$. Všechny vrcholy $v$, které mají
		$D(v)=0$ přidej do fronty.
		\item Dokud není fronta prázdná vezmi a vypiš vrchol $v$ z čela fronty a pro každou
		hranu směřující z něho do $w$ proveď $D(w)-=1$, pokud nyní $D(w)==0$, zařaď ho do fronty.
	\end{itemize}
	\normalsize
}

\card[Vlastnosti]{Řadící algoritmy}{
	\begin{description}
		\item[Pamětová náročnost] Rozlišují se In-place a Out-of-place algoritmy.
		\item[Stabilita] Stabilní, pokud správně seřazené prvky ze vstupu mají stejné pořadí i na výstupu.
		\item[Citlivost] Určuje, jestli se mění časová složitost na základě vstupu.
	\end{description}
}

\end{center}
\end{document}
